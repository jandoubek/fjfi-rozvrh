\documentclass[a4paper, 11pt]{article}
\usepackage[czech]{babel}
\usepackage{latexsym}
\usepackage[utf8]{inputenc}	%znaková sada utf8
\usepackage{indentfirst}

\renewcommand{\arraystretch}{1.5}
\renewcommand{\familydefault}{\sfdefault}

%\renewcommand{\thechapter}{\Alph{chapter}}


\oddsidemargin=-10mm   % levý okraj větší (kvůli vazbě)
\topmargin=-25mm      % horní okraj trochu menší
\textwidth=180mm      % šířka textu na stránce
\textheight=270mm     % výška textu na stránce

\pagenumbering{arabic} % číslování stránek arabskými číslicemi
\pagestyle{plain}      % stránky číslované dole uprostřed

\parindent=0pt % odsazení 1. řádku odstavce
\parskip=7pt   % mezera mezi odstavci
\frenchspacing % aktivuje použití některých českých typografických pravidel

% definice marka pro české uvozovky:
\def\bq{\mbox{\kern.1ex\protect\raisebox{-1.3ex}[0pt][0pt]{''}\kern-.1ex}}
\def\eq{\mbox{\kern-.1ex``\kern.1ex}}
\def\ifundefined#1{\expandafter\ifx\csname#1\endcsname\relax }%
\ifundefined{uv}%
        \gdef\uv#1{\bq #1\eq}
\fi
% konec .... použití makra pro psaní českých uvozovek: \uv{text uvnitř uvozovek}
\title{Schéma použití webové aplikace}
\date{\today}

\begin{document}
\maketitle
\section{Definice pojmů}
\begin{description}
\item[Lekce] --- jednoznačně určená výuková hodina v čase, tj. předmět, katedra (barva), vyučující, místnost, čas (od + do případně od + doba trvání), (ne)pravidelnost výuky.
\end{description}

\part{Vyhledávání lekcí}
Obecným požadavkem u vyhledávání lekcí je výstup: seznam lekcí vyhovujících zadanému filtru. Tento seznam nemusí být nutně setříděný.

\section{Student}
V této sekci definuji několik nejčastějších požadavků na službu z pohledu studenta.

\subsection{Rozvrh podle kruhu}
V tomto schématu je pro studenta důležité získat na co nejmenší počet kliknutí celý svůj rozvrh. Mezi studenty půjde asi o nejčastěji využívanou funkcionalitu, protože jsou buď tak dobří, že změny nepotřebují, nebo změny potřebují, ale je pro ně nejrychlejší z/do tohoto rozvrhu ubrat/přidat předmět.

\subsubsection{Strategie}
Student volí pouze ročník, zaměření a kruh, tato trojice jednoznačně určuje jeho požadovaný rozvrh.

\subsubsection{Požadavky}
Vzniká požadavek na filtrování podle ročníku, zaměření a kruhu.

\subsection{Konkrétní předmět}
Student chce zjistit, kdy probíhá výuka (z různých důvodů) vybraného předmětu. Zde může nastat několik situací:
\begin{itemize}
\item Student zná kromě názvu předmětu i další informace,
\item Student zná pouze název předmětu.
\end{itemize}

\subsubsection{Strategie: název + další informace o předmětu}
Použitím filtru a zadáním známých údajů co nejblíže specifikuje hledaný předmět. Čím více parametrů bude znát, respektive čím více filtrů bude moct aplikovat, tím více zmenší výslednou množinu vyhovujících lekcí.

\subsubsection{Požadavky}
Filtrování podle vlastností definovaných pro lekci, není ale jasné, které všechny jsou nutné. Potřebujeme pouze získat seznam lekcí vyhovujících filtru.

\subsubsection{Strategie: pouze název předmětu}
Student hledá podle názvu (či zkratky) předmětu, případně je nucen dohledat si nějaké informace o předmětu, pokud nechce vybírat ze všech lekcí, tj. není aplikován žádný filtr.

\subsubsection{Požadavky}
Zde vzniká požadavek na filtrování podle názvu/zkratky předmětu. Není nutné nabízet studentovi seznam všech předmětů, ale třeba použít vyhledávací pole.

\subsection{Konkrétní čas}
Toto schéma nastává například v případě, že student už má k dispozici nějakou verzi rozvrhu a rád by zaplnil mezery mezi jednotlivými lekcemi.

\subsubsection{Strategie}
Tato strategie je velice jednoduchá. Student potřebuje pouze zadat čas od/do a vybrat si mezi nabídnutými lekcemi.

\subsubsection{Požadavky}
Vzniká požadavek na filtrování podle času.

\section{Vyučující}
Nyní se podívejme na vytváření rozvrhů z role vyučujícího.

\subsection{Vlastní rozvrh}
Vyučujícího zajímá pouze jeho osobní rozvrh.

\subsubsection{Strategie}
Vyučující má k dispozici své jméno, což je nejjednodušší možnost, jak získat svůj rozvrh.

\subsubsection{Požadavky}
Vzniká požadavek na filtrování podle jména vyučujícího.

\subsection{Rozvrh vyučovaného předmětu}
Tento případ kopíruje strategii již zmiňovaného studentova vyhledávání konkrétního předmětu. U vyučujícího můžeme předpokládat, že kromě názvu předmětu zná i další informace.

\subsubsection{Požadavky}
Požadavek na filtrování podle jména předmětu není nutný, ale jde jistě o velmi pohodlnou možnost.

\section{Zhodnocení požadavků}
V předchozích schématech vznikly nutné požadavky na vyhledávání dle ročníku, zaměření, kruhu, předmětu, času. Pro pohodlnost doporučuji zahrnout i zbylé parametry lekcí. 

\part{Editace rozvrhu}
V této části již nevidím rozdíl v přístupu studenta či vyučujícího, 
oba budu tedy shodně nazývat uživateli. Ve strategiích zároveň předpokládám, že akce, které bude uživatel vykonávat po vyfiltrování, se budou týkat jen a pouze přidávání lekci, nikoli například dalších úprav filtru, či jiné.

\section{Přidání lekcí}
Uživatel má po vyhledávání lekcí k dispozici seznam lekcí vyhovujících filtru a chce přidat do stávajícího rozvrhu (i prázdného) jednu, více nebo všechny lekce.

\subsection{Přidání všech lekcí}
Uživatel použil takový filtr, který mu vrátil seznam lekcí, z nichž chce všechny přidat do svého rozvrhu.

\subsubsection{Strategie}
Uživatel nechce žádným způsobem vybírat (označovat) jednotlivé lekce:
\begin{itemize}
\item Obecně jich je mnoho,
\item Chce prostě všechny.
\end{itemize}

\subsubsection{Požadavky}
Vzniká požadavek na funkci \uv{přidat vše}, která, jak už název napovídá, přidá všechny vybrané lekce do stávajícího rozvrhu.

\subsection{Přidání jedné lekce}
Uživatel použil takový filtr, který mu vrátil seznam lekcí, z nichž pouze jednu chce přidat do svého rozvrhu.

\subsubsection{Strategie}
Uživatel ví, kterou lekci z vyfiltrovaného seznamu chce. Nepotřebuje ji nijak označovat, pouze jednoduchým úkonem přidat do rozvrhu.

\subsubsection{Požadavky}
Vzniká požadavek na přidání jedné z více lekcí.

\subsection{Přidání více lekcí}
Uživatel použil takový filtr, který mu vrátil seznam lekcí, z nichž pouze některé, ale obecně více než jednu, chce přidat do svého rozvrhu.

\subsubsection{Strategie}
Uživatel musí ze seznamu vybrat ty, které chce přidat do rozvrhu.

\subsubsection{Požadavky}
Vzniká požadavek na přidání více lekcí z mnoha.

\section{Odebrání lekcí}
Podobně jako v předchozí části může uživatel chtít odebrat z již hotového rozvrhu jednu, více či všechny lekce. Strategie i požadavky vesměs kopírují své protějšky v předchozí sekci.

\subsection{Odebrání všech lekcí}
Uživatel má k dispozici takový rozvrh, který mu nevyhovuje do té míry, že se rozhodl jej úplně vymazat.

\subsubsection{Strategie}
Uživatel nechce žádným způsobem vybírat (označovat) jednotlivé lekce:
\begin{itemize}
\item Obecně jich je mnoho,
\item Chce prostě odstranit všechny.
\end{itemize}

\subsubsection{Požadavky}
Vzniká požadavek na smazání všech lekcí v rozvrhu.

\subsection{Odebrání jedné lekce}
Uživateli v aktuálním rozvrhu nevyhovuje jedna z lekcí a chce ji odstranit

\subsubsection{Strategie}
Uživatel ví, kterou lekci z rozvrhu chce odstranit. Nepotřebuje ji nijak označovat, pouze jednoduchým úkonem odstranit.

\subsubsection{Požadavky}
Vzniká požadavek na odebrání jedné z více lekcí.

\subsection{Odebrání více lekcí}
Uživateli rozvrh částečně nevyhovuje, nechce jej smazat celý a zároveň chce smazat více lekcí.

\subsubsection{Strategie}
Uživatel musí z rozvrhu vybrat ty, které chce z rozvrhu odebrat.

\subsubsection{Požadavky}
Vzniká požadavek na odebrání více lekcí z mnoha.

\section{Zhodnocení požadavků}
Vznikly požadavky na přidání/odebrání jednoho, více nebo všech předmětů.  Podaří-li se implementovat funkci přidání/odebrání jednoho předmětu tak, že bude velmi jednoduchá a rychlá, mohla by být využita i pro přidání/odebrání více předmětů. Otázkou k debatě je, jestli uživatel nebude chtít ze seznamu vybírat (označovat/odznačovat) jednotlivé lekce a až potom je přidat/odebrat.

\part{Práce s hotovým rozvrhem}
V této části má uživatel již rozvrh hotový a rád by si jej uložil, případně si jej již uložil, ale rozhodl se jej znovu editovat.

\section{Export rozvrhu}
Uživatel použil kombinaci vyhledávání lekcí a editace rozvrhu k získání vyhovujícího rozvrhu a chce si jej uložit a získat pro sebe.

\subsection{Strategie}
Uživatel si chce jednoduchých způsobem uložit stávající rozvrh.

\subsection{Požadavky}
Vzniká požadavek na výběr formátu. Vzniká požadavek na export rozvrhu ve vybraném formátu.

\section{Import rozvrhu}
Uživatel má rozvrh uložen na svém počítači. Buď si jej v minulosti sám vypracoval a uložil, případně jej získal od někoho. Nyní se rozhodl pro jeho změnu.

\subsection{Strategie}
Uživatel chce nahrát svůj rozvrh do aplikace a ten následně upravit dle svých představ.

\subsection{Požadavky}
Vzniká požadavek na import rozvrhu do aplikace a umožnění další práce s ním.

\section{Zhodnocení požadavků}
Vznikají požadavky na import a export rozvrhu. Vzniká požadavek na výběr formátu exportovaného rozvrhu.

\end{document}