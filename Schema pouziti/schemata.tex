\documentclass[a4paper, 11pt]{article}
\usepackage[czech]{babel}
\usepackage{latexsym}
\usepackage[utf8]{inputenc}	%znaková sada utf8
\usepackage{indentfirst}

\renewcommand{\arraystretch}{1.5}
\renewcommand{\familydefault}{\sfdefault}

%\renewcommand{\thechapter}{\Alph{chapter}}


\oddsidemargin=-10mm   % levý okraj větší (kvůli vazbě)
\topmargin=-25mm      % horní okraj trochu menší
\textwidth=180mm      % šířka textu na stránce
\textheight=270mm     % výška textu na stránce

\pagenumbering{arabic} % číslování stránek arabskými číslicemi
\pagestyle{plain}      % stránky číslované dole uprostřed

\parindent=0pt % odsazení 1. řádku odstavce
\parskip=7pt   % mezera mezi odstavci
\frenchspacing % aktivuje použití některých českých typografických pravidel

% definice marka pro české uvozovky:
\def\bq{\mbox{\kern.1ex\protect\raisebox{-1.3ex}[0pt][0pt]{''}\kern-.1ex}}
\def\eq{\mbox{\kern-.1ex``\kern.1ex}}
\def\ifundefined#1{\expandafter\ifx\csname#1\endcsname\relax }%
\ifundefined{uv}%
        \gdef\uv#1{\bq #1\eq}
\fi
% konec .... použití makra pro psaní českých uvozovek: \uv{text uvnitř uvozovek}
\title{Schéma použití webové aplikace}
\date{\today}

\begin{document}
\maketitle
\section{Definice pojmů}
\begin{description}
\item[Lekce] --- jednoznačně určená výuková hodina v čase, tj. předmět, katedra (barva), vyučující, místnost, čas (od + do případně od + doba trvání), (ne)pravidelnost výuky.
\end{description}

\part{Vyhledávání lekcí}
Obecným požadavkem u vyhledávání lekcí je výstup: seznam lekcí vyhovujících zadanému filtru. Tento seznam nemusí být nutně setříděný.

\section{Student}
V této sekci definuji několik nejčastějších požadavků na službu z pohledu studenta.

\subsection{Rozvrh podle kruhu}
V tomto schématu je pro studenta důležité získat na co nejmenší počet kliknutí celý svůj rozvrh. Mezi studenty půjde asi o nejčastěji využívanou funkcionalitu, protože jsou buď tak dobří, že změny nepotřebují, nebo změny potřebují, ale je pro ně nejrychlejší z/do tohoto rozvrhu ubrat/přidat předmět.

\subsubsection{Strategie}
Student volí pouze ročník, zaměření a kruh, tato trojice jednoznačně určuje jeho požadovaný rozvrh.

\subsubsection{Požadavky}
Vzniká požadavek na filtrování podle ročníku, zaměření a kruhu.

\subsection{Konkrétní předmět}
Student chce zjistit, kdy probíhá výuka (z různých důvodů) vybraného předmětu. Zde může nastat několik situací:
\begin{itemize}
\item Student zná kromě názvu předmětu i další informace,
\item Student zná pouze název předmětu.
\end{itemize}

\subsubsection{Strategie: název + další informace o předmětu}
Použitím filtru a zadáním známých údajů co nejblíže specifikuje hledaný předmět. Čím více parametrů bude znát, respektive čím více filtrů bude moct aplikovat, tím více zmenší výslednou množinu vyhovujících lekcí.

\subsubsection{Požadavky}
Filtrování podle vlastností definovaných pro lekci, není ale jasné, které všechny jsou nutné. Potřebujeme pouze získat seznam lekcí vyhovujících filtru.

\subsubsection{Strategie: pouze název předmětu}
Student hledá podle názvu (či zkratky) předmětu, případně je nucen dohledat si nějaké informace o předmětu, pokud nechce vybírat ze všech lekcí, tj. není aplikován žádný filtr.

\subsubsection{Požadavky}
Zde vzniká požadavek na filtrování podle názvu/zkratky předmětu. Není nutné nabízet studentovi seznam všech předmětů, ale třeba použít vyhledávací pole.

\subsection{Konkrétní čas}
Toto schéma nastává například v případě, že student už má k dispozici nějakou verzi rozvrhu a rád by zaplnil mezery mezi jednotlivými lekcemi.

\subsubsection{Strategie}
Tato strategie je velice jednoduchá. Student potřebuje pouze zadat čas od/do a vybrat si mezi nabídnutými lekcemi.

\subsubsection{Požadavky}
Vzniká požadavek na filtrování podle času.

\section{Vyučující}
Nyní se podívejme na vytváření rozvrhů z role vyučujícího.

\subsection{Vlastní rozvrh}
Vyučujícího zajímá pouze jeho osobní rozvrh.

\subsubsection{Strategie}
Vyučující má k dispozici své jméno, což je nejjednodušší možnost, jak získat svůj rozvrh.

\subsubsection{Požadavky}
Vzniká požadavek na filtrování podle jména vyučujícího.

\subsection{Rozvrh vyučovaného předmětu}
Tento případ kopíruje strategii již zmiňované studentovo vyhledávání konkrétního předmětu. U vyučujícího můžeme předpokládat, že kromě názvu předmětu zná i další informace.

\subsubsection{Požadavky}
Požadavek na filtrování podle jména předmětu není nutný, ale jde jistě o velmi pohodlnou možnost.

\subsection{Zhodnocení požadavků}
V předchozích schématech vznikly nutné požadavky na vyhledávání dle ročníku, zaměření, kruhu, předmětu, času. Pro pohodlnost doporučuji zahrnout i zbylé parametry lekcí. 

\part{Editace rozvrhu}

\part{Export rozvrhu}

\end{document}